\section{Creating PEBBL Applications}

When using PEBBL, you should generally first create a serial PEBBL
application and debug it using whatever C++ development tools
you are most comfortable with.  Then you should extend it to a
parallel application using MPI.

\subsection{Compiling and linking your application}
In developing a new PEBBL application, you may either keep your source
files outside the \texttt{acro-pebbl} directory tree, or incorporate
them into the tree.  If you keep your files outside, you should
inspect the results of the PEBBL \texttt{make} command or the contents
of \texttt{acro-pebbl/test/build.out}, and use compatible commands to
compile and link your application source files.  In particular, your
compile commands' \texttt{\#include} path should contain the directory
\texttt{acro-pebbl/include}, and your link commands should include the
libraries \texttt{acro-pebbl/lib/libpebbl.a} and
\texttt{acro-pebbl/lib/libutilib.a}.  In standard unix/Linux
environments, for example, you could use the compiler/linker directives
\begin{codeblock}
-I\textrm{[\emph{path}}/\textrm{]}acro-pebbl/include
-L\textrm{[\emph{path}}/\textrm{]}acro-pebbl/lib -lpebbl -lutilib
\end{codeblock}
An alternative approach is to place your application source and header
files within the PEBBL directory tree and use PEBBL's own compilation
tools to compile your code.  The recommended location for both your
source and header files is \url{acro-pebbl/src/example/pebbl}.  To
incorporate your files into PEBBL's build process, you should proceed
as follows:
\begin{enumerate}
\item Modify the file \url{acro-pebbl/src/Makefile.am}
  to include your new source files in a pattern similar to those for
  the \texttt{knapsack} application.  
\item Incorporate any
  new header files you have created into the list contained in the
  file \url{acro-pebbl/src/headers/Makefile.am}.  
\item Rerun the configuration and compilation procedure of
  Section~\ref{sec:compile}.  The \texttt{configure} step, guided by
  your altered \texttt{Makefile.am} files, will incorporate your
  application into PEBBL's build process.
\end{enumerate}

\subsection{Defining a serial application}
\label{sec:serialMethods}
To create a basic serial PEBBL application, you need to:
\begin{itemize}
\item Define a class extending \texttt{branching}
\item Define a class extending \texttt{branchSub}
\item If necessary, create a class extending \texttt{solution}
\item Create a ``driver'' program that runs your algorithm.
\end{itemize}
You may concentrate your code for all these purposes in a single C++
file.  Conventionally, however, you would use a header (\texttt{.h} or
\texttt{.hpp}) file to define your new classes, one or more additional
C++ source files containing code to implement methods for those
classes, and a separate C++ source file containing the driver.

To load the basic definitions of \texttt{branching},
\texttt{branchSub}, and \texttt{solution} 
into the compiler, you should \texttt{\#include}
the file \texttt{<pebbl/branching.h>}.  Supposing your classes are
called \texttt{\emph{myBranching}} and \texttt{\emph{myBranchSub}},
and you are using a single header file, it should take the general
form shown in Figure~\ref{fig:code1}.

\begin{figure}[tbp]
\begin{center}
\fbox{
\begin{minipage}{0.96\textwidth}
\vspace{1ex}
\begin{codeblock}
\#include <pebbl/branching.h> \\
\\
using namespace pebbl; \\
\\
class \emph{myBranchSub};  // Forward declaration \\
\\
clas\=s \emph{myBranching} : virtual public branching \\
\{\\
\>friend class \emph{myBranchSub}; \\
\\
public:\\
\>$\quad\vdots$ \\
\>\textit{myBranching\textrm{ methods and data here}} \\
\>$\quad\vdots$ \\
\} \\
\\
class \emph{myBranchSub} : virtual public branchSub \\
\{\\
\>friend class \emph{myBranching}; \\
\\
protected:\\
\>// A pointer to the global branching object\\
\> \emph{myBranching}* globalPtr;\\
$\quad\vdots$ \\
public:\\
\>  // Return a pointer to the global branching object\\
\>  \emph{myBranching}* global() const \{ return globalPtr; \}\\
\\
\>  // Return a pointer to the base class of the global branching object \\
\>  branching* bGlobal() const \{ return global(); \}\\
\>$\quad\vdots$ \\
\>\textit{myBranchSub\textrm{ methods and data here}} \\
\>$\quad\vdots$ \\
\}
\end{codeblock}
\vspace{0.2ex}
\end{minipage}
}
\end{center}
\vspace{-2ex}
\caption{Standard code pattern for creating a serial PEBBL application.}
\label{fig:code1}
\end{figure}

We now describe how to construct these classes; in the course of this
description, you may also refer to the files \texttt{serialCore.h},
and \texttt{serialKnapsack.\{h,cpp\}}
%%
%% WEH TODO - INSERT SEMI_ASSIGNMENT PROBLEM
%%
%%, and \textbf{insert the semi-assignment example} 
%%
in the directory \url{acro-pebbl/packages/pebbl/src/example/pebbl}.

\subsubsection{Methods you should create ---
  \texttt{branching}-derived class}

In your \texttt{branching}-derived derived class, for example
\texttt{\emph{myBranching}}, you should define the following methods:

\boxheader{Constructor}
Classes derived from \texttt{branching} should have a constructor with
no arguments; the diamond inheritance pattern used by PEBBL means that
constructors with arguments are in general not advisable for 
\texttt{branching}- and \texttt{branchSub}-derived classes.  The
constructor should contain a call to
\texttt{branching::\linebreak[0]branchingInit($\cdots$)}. If you want
to maximize the objective function, you should supply
\texttt{branchingInit} an argument of \texttt{maximization}; if you
want to minimize, you can leave the argument list blank, or supply the
argument \texttt{minimization}.  Thus, one might have something like
\begin{codeblock}
\emph{myBranching}()  \\
\{ \\
~~~\=branchingInit(minimization); \\
\>~~$\vdots$ \\
\}; 
\end{codeblock}
The direction of minimization is stored in the data member
\texttt{branching::sense}.  The values \texttt{minimization} and
\texttt{maximization} are C++ \texttt{enum} values, equal to $+1$ and
$-1$, respectively.

\boxheader{Destructor} 
Naturally, you should also have a destructor for your
\texttt{branching}-derived class, for example:
\texttt{$\sim$\emph{myBranching}() \{ $\cdots$ \};}.

\method{branchSub* blankSub()}
You must provide a \texttt{blankSub()} method of return type
\texttt{branchSub*} that returns an empty subproblem specific to your
application.  To avoid circularity, the class declaration for
\texttt{\emph{myBranching}} should declare this method, but not
include its code, as in:
\begin{codeblock}
branchSub* blankSub();
\end{codeblock}
Once
\texttt{\emph{myBranching}} and
\texttt{\emph{myBranchSub}} are both declared, you should have the
actual code, for example:
\begin{codeblock}
bra\=nchSub* \emph{myBranching}::blankSub() \\
\{ \\
\>   \emph{myBranchSub}* newSP = new \emph{myBranchSub}; \\
\>    newSP->setGlobalInfo(this);\\
\>    return newSP; \\
\};
\end{codeblock}
Here, \texttt{\emph{myBranchSub}::setGlobalInfo} is a method that is
required to be a member of the subproblem class (see
Section~\ref{sec:bsubmethods} below), and
copies any necessary information from the 
\texttt{\emph{myBranching}} object into a \texttt{\emph{myBranchSub}}
object.  At a minimum, it should set the subproblem's
\texttt{globalPtr} member.  

\method{bool setupProblem(int\& argc,char**\& argv)}
This method is responsible for reading in any input data describing
the problem instance.  Its arguments \texttt{argc} and \texttt{argv}
are standard unix-style command line descriptors.  However, by the
time \texttt{setupProblem} is called, these arguments will be
``cleaned'' of any MPI-related information and parameter settings
recognized by PEBBL.  Typically, \texttt{setupProblem} should simply
read a problem description from the file whose name is pointed to 
by \texttt{argv[1]}.  The \texttt{setupProblem} method should return
\texttt{true} is problem setup was successful.

%% \method{serialPrintSolution(const char* header,const char*
%%   footer,std::ostream\& s)} 
%% Typically, the \texttt{branching}-derived class should have the
%% ability to store a single solution to the optimization problem.  The
%% \texttt{serialPrintSolution} method should print this solution to the
%% stream \texttt{s}, preceded by the \texttt{header}, and followed by
%% \texttt{footer}.  Strictly speaking, you do not have to implement this
%% method --- it has a default implementation of \texttt{\{~\}}.
%% However, it should be defined if you need PEBBL to print
%% out anything besides the \emph{value} of the solution.

\method{void reset(bool resetVB = true)} Perform any initializations
needed after \texttt{setupProblem} and before solving the actual
problem.  For some applications, the default implementation
\texttt{branching::reset} may suffice.  The optional argument
\texttt{resetVB} indicates whether virtual base classes should be
reset as well; it is present to allow one to avoid multiple redundant
initializations of such bases classes.  Typically, it can be ignored,
although there may be a minor performance penalty.  If you override the
\texttt{reset} method, make sure your implementation invokes
\texttt{branching::reset} in addition to any application-specific
initializations it performs.

\subsubsection{Methods you should create ---
  \texttt{branchSub}-derived class}
\label{sec:bsubmethods}

\boxheader{Constructor} 
As with the \texttt{branching}-derived class,
you should have an empty-argument constructor for your
\texttt{branchSub}-derived class, for example:
\begin{codeblock}
\emph{myBranchSub}() \{  $\cdots$ \}; \\
\end{codeblock}

\boxheader{Destructor} 
A destructor is naturally required, for example:
\begin{codeblock}
$\sim$\emph{myBranchSub}() \{ $\cdots$ \}; \\
\end{codeblock}

\method{branching* bGlobal() const}
This method is required for methods in \texttt{branchSub} to locate
the corresponding problem-wide information in the corresponding
\texttt{branching} object.  Presuming you have declared a data member
\texttt{\emph{myBranching}* globalPtr} as in Figure~\ref{fig:code1}, 
then Figure~\ref{fig:code1}'s definition should suffice:
\begin{codeblock}
branching* bGlobal() const \{ return global(); \};
\end{codeblock}
Your implementations of methods such as \texttt{boundComputation()}
will almost certainly require access to problem-wide but
application-specific information.  In the implementation above, this
may be obtained directly through \texttt{globalPtr} or slightly more
elegantly via
\begin{codeblock}
\emph{myBranching}* global() const \{ return globalPtr; \};
\end{codeblock}

\method{void setGlobalInfo(\emph{myBranching*} global\_)}
The purpose of this routine should be to ``bind'' a particular
subproblem object to the problem description embodied in the object
\texttt{*global\_}.  At a minimum, this requires setting
\texttt{globalPtr}.  For example:
\begin{codeblock}
voi\=d setGlobalInfo(\emph{myBranching*} global\_) \\
\{ \\
\> globalPtr = global\_; \\
\>\textrm{(Copy any other desired information from
  \texttt{*global\_} )}\\
\}
\end{codeblock}


\method{void setRootComputation()}
A call to this method indicates that a \texttt{\emph{myBranchSub}}
object should be made into a root subproblem.  Information on the
problem instance should be obtained, for example, via
\texttt{global()} or \texttt{globalPtr}.

\pagebreak[1]

\method{void boundComputation()} 
This method should attempt to bound the subproblem.  When the bound
computation is complete, it should execute \texttt{setState(bounded)}.
If the method exits without calling \texttt{setState(bounded)} to
declare the subproblem \texttt{bounded}, PEBBL will assume that the
bounding operation is incomplete: it will call
\texttt{boundComputation()} repeatedly (but perhaps not immediately) 
until the subproblem is
fathomed or declared \texttt{bounded}.  To indicate the value of the
bound, you set the \texttt{double} data member \texttt{bound};
you may update \texttt{bound} as many times as you like, and PEBBL
will use the information immediately, even before the
problem state becomes \texttt{bounded} --- for example, a sequence of calls
to \texttt{boundComputation()} may gradually improve the bound.
To indicate infeasibility of a subproblem, or that it cannot possibly
improve upon the current solution stored in
\texttt{bGlobal()->incumbentValue}, you may execute
\texttt{setState(dead)}. 


\method{bool candidateSolution()} PEBBL will only invoke this method
for subproblems in the \texttt{bounded} state.  Returning
\texttt{true} means that the just-computed bound is \emph{exact}: not
only is it a bound on all solutions in the region of the search space
corresponding to the subproblem, but that bound is also attained by at
least one feasible solution in that region.  Unless one is enumerating
multiple solutions, such a subproblem is considered a \emph{terminal}
node of the branch-and-bound tree, and will not be separated.  In
integer programming, for example, you would want
\texttt{candidateSolution()} to return \texttt{true} whenever the
linear programming relaxation produces a solution that has all integer
values.  Returning \texttt{true} also indicates to PEBBL that you have
computed a feasible solution to the overall optimization problem, as
opposed to just a bound.  PEBBL will most likely invoke
\texttt{extractSolution()} (see below) to retrieve this solution.


\method{solution* extractSolution()} PEBBL will only call this method
for \texttt{bounded} subproblems for which \texttt{candidateSolution}
has returned \texttt{true}.  The method should return a pointer to a
\texttt{solution}-derived object containing a solution corresponding
to the subproblem bound.  See Section~\ref{sec:solclassdetails} for
more information on \texttt{solution} objects.  Ownership of the
returned object is ceded to PEBBL.


\method{int splitComputation()} 
This method should attempt to separate the subproblem.  Similarly to
\texttt{boundComputation()}, it will be invoked repeatedly until it
calls \texttt{setState(separated)}.  The value returned is the number $k$
of child problems generated (simply $2$ for many applications).  The
number of children can vary between subproblems.
If you return without declaring the problem \texttt{separated}, PEBBL
will ignore the return value and invoke \texttt{splitComputation()}
again later.  In the course of separating the problem, you may update
\texttt{bound} at any time to reflect any further information gained
in the course of separation; PEBBL will use this
information immediately.  This routine may also execute
\texttt{setState(dead)}, in which case PEBBL will immediately prune
the subproblem.

If PEBBL is not in enumeration mode, it will call
\texttt{splitComputation()} only for nodes for which
\texttt{candidateSolution} has returned \texttt{false}.  On the other
hand, if PEBBL is enumerating multiple solutions, it is possible that
\texttt{splitComputation()} may be called for nodes for which
\texttt{candidateSolution} has returned \texttt{true}.  If you need
your application to be compatible with enumeration, it should be able
to handle this case of separating an otherwise ``terminal''
subproblem.  In some applications, such
nodes may require a completely different separation technique than
other nodes, and possibly a different number of children.  Ideally,
when separating a subproblem $P$ for which \texttt{candidateSolution}
has returned \texttt{true}, you should create one or more children
corresponding to search-space regions which are disjoint and contain
all solutions corresponding to $P$ \emph{except} the one just returned
by $P\!$\texttt{.extractSolution()}.  This latter solution will already
be known to PEBBL and need not be examined again.  If a node is
``truly terminal'' in that it corresponds to a subset of the search
space containing only a single distinct feasible solution, then
\texttt{splitComputation()} should call \texttt{setState(dead)} and
return $0$.

\method{branchSub* makeChild(int whichChild)} 
This method should create a child problem and return a pointer to it.  
The argument
\texttt{whichChild} may take any value between $0$ and $k-1$, where $k$
was the value returned by \texttt{splitComputation()} for this
subproblem.  
A value of $0$ indicates you should create the first child, a
value of $1$ indicates you should create the second child, and so
forth.  

When you create a child problem, you should make sure that its
\texttt{globalPtr} is set as in the parent problem, and all local
data are initialized correctly.  In order for PEBBL to correctly track
subproblem depth and counts of created subproblems, it is also necessary to
correctly set up various data members of the \texttt{branchSub} class.  PEBBL
provides a method called \texttt{branchSubAsChildOf} to perform this
operation.  Specifically, the child problem should invoke
\texttt{branchSubAsChildOf(parent)} soon after it is created, where
\texttt{parent} is a pointer to the parent subproblem.  Alternatively, the parent
subproblem may call
\begin{codeblock}
\emph{child}->branchSubAsChildOf(this);
\end{codeblock}
where \emph{\texttt{child}} is a pointer to the child subproblem just created.


%% \method{void updateIncumbent()}
%% This method should set the state of the \texttt{branching}-derived
%% class --- \emph{e.g. \texttt{myBranching}} --- to reflect a feasible
%% problem solution correspondig to this 
%% subproblem.  PEBBL will typically call this method
%% after \texttt{candidateSolution()} returns \texttt{true}.  A typical
%% implementation will look like:
%% \begin{codeblock}
%% voi\=d \emph{myBranchSub}::updateIncumbent() \\
%% \{\\
%% \>bGlobal()->incumbentValue = bound;  // Set the incumbent value \\
%% \>\textrm{(Copy the solution itself to \texttt{*global()})} \\
%% \>bGlobal()->signalIncumbent(); \\
%% \}
%% \end{codeblock}
%% Once the solution has been copied to a \texttt{branching}-derived
%% class, it can be written by the \texttt{serialPrintSolution($\cdots$)}
%% method.  The call to \texttt{signalIncumbent()} tells PEBBL that
%% the subproblem pool should be pruned.  Once the application is ported
%% to the parallel layer, \texttt{signalIncumbent()} will also initiate
%% interprocessor communication to broadcast the new incumbent value.


\subsubsection{The \texttt{solution} class and related methods}
\label{sec:solclassdetails}
PEBBL manipulates feasible problem solutions via objects whose type is
derived from \texttt{solution}.  For example, the
\texttt{extractSolution()} method should return a \texttt{solution*},
as should the \texttt{initialGuess} method; see
Section~\ref{sec:seradd} below.  The \texttt{solution} class itself
contains (among others) the following members:
\begin{description}
\item{\texttt{double value}}: the solution's objective value.
\item{\texttt{int serial}}: a serial number (assigned automatically by
  PEBBL).
\item{\texttt{size\_type hashValue}}: a value for use in hash tables
\end{description}
It is possible to represent solutions via the \texttt{solution} base
class; however, such a representation will essentially contain only
the solution's objective value, and no other information.  
To construct such an object, use the
constructor \texttt{solution(branching*)}, and then set the
\texttt{value} of the resulting object.  For example, for the code in the
\texttt{\emph{myBranchSub}} class of Figure~\ref{fig:code1}, one would
use
\begin{codeblock}
s = new solution(global()); \\
s.value = \emph{objective value};
\end{codeblock}
Far more likely, however, you will want a true solution representation
that contains more information than just the objective value.  In this
case, you should use another class derived from \texttt{solution}.  Here,
there are two choices: use the template class
\texttt{arraySolution<\emph{T}>}, or derive your own \texttt{solution}-based
class.

Note that each instance of the \texttt{solution} class contains its
own \texttt{sense} member which must be set to \texttt{maximization}
or \texttt{minimization} consistently with \texttt{branching}-derived
object describing the problem instance (for example,
\texttt{\emph{myBranching}}).  Typically, this is handled by passing a
\texttt{branching*} pointer to the \texttt{solution} constructor.

Starting with PEBBL 1.6, the \texttt{solution} class includes a reference
counter, and its destructor will throw an exception if activated when this
counter is nonzero.  One should not directly invoke the destructors of
\texttt{solution}-derived objects by using the
\texttt{delete} operator, but instead use the \texttt{dispose()} method of the
\texttt{solution} class.  This method decrements the reference counter and then
call the destructor if the counter is zero.  The reference counter may also be
incremented or decremented by the \texttt{incrementRefs()} and
\texttt{decrementRefs()} methods, respectively.  If a
\texttt{solution}-derived object is allocated on the stack by direct
declaration, rather than by using the \texttt{new} operator, one should call
its \texttt{decrementRefs()} method before it passes out of scope; otherwise
its destructor will trigger an exception.

The template \texttt{arraySolution<\emph{T}>} is a ready-made PEBBL-supplied
class that represents a solution as a one-dimensional array (that is,
a vector) of some type \texttt{\emph{T}} which may be implicitly converted to
a \texttt{double} without loss of information.  To create such a
solution object, invoke a constructor (supplied by PEBBL) of the form
\texttt{arraySolution<\emph{T}>($z$,$v$,$b$,$t$,$\nu$)}, where:
\begin{quote}
\begin{description}
\item[$z$] denotes the objective value of the solution 
\item[$v$] is either a UTILIB array object of type
  \texttt{BasicArray<\emph{T}>}, an STL vector of type 
  \texttt{vector<\emph{T}>}, or
  some object derived from either of those possibilities (for example,
  the UTILIB \texttt{IntVector} type derives from
  \texttt{BasicArray<int>}, and may thus be used to create an
  \texttt{arraySolution<int>}).  This information is copied into the
  new \texttt{arraySolution} object.
\item[$b$] is a pointer
  to the \texttt{branching}-derived object describing the 
  problem instance; within the class \texttt{\emph{myBranchSub}} of
  Figure~\ref{fig:code1}, you would set $b =
  \text{\texttt{global()}}$, and within the
  \texttt{\emph{myBranching}} of Figure~\ref{fig:code1}, you would set
  $b = \text{\texttt{this}}$.  One function of this argument is to
  make sure that the maximization/minimization \texttt{sense} of the
  \texttt{arraySolution} is set correctly.
\item[$t$] is optional, and is simply a \texttt{const char*} pointing to a
  description of the type of solution, for example, \texttt{"Vector
    of integers"}. 
\item[$\nu$] is also optional, and is a pointer to a
  \texttt{BasicArray<CharString>} object containing names of the
  decision variables.  It should be at least as long as $v$.  These
  names are used when printing the solution contents.  
\end{description}
\end{quote}
For descriptions of the types \texttt{BasicArray} and
\texttt{CharString}, see the UTILIB documentation.

If \texttt{arraySolution} does not apply to your problem, is
inconvenient, or will be inefficient, you should derive your own
class of objects from \texttt{solution}, as in:
\begin{codeblock}
class \emph{mySolution} : virtual public solution \{ $\cdots$ \}
$\;.$
\end{codeblock}
You may use more than one kind of solution class in a single
application.  

To facilitate hashing and comparison, solutions normally
have a ``sequence representation'', as described in
Section~\ref{sec:solclass}.  This representation is essentially a
one-dimensonal array of \texttt{double}s, although it need not be (and
probably should not be) explicitly stored.  By default, PEBBL uses
this sequence representation to compute a solution's
\texttt{hashValue} and to compare two solutions to detect duplicates.
The solutions are considered duplicates if and only if their sequence
representations are identical in both length and contents.

In addition to application-specific data members describing a
solution, a customized solution class like \texttt{\emph{mySolution}}
should typically define the following \texttt{virtual} methods:

\vspace{2ex}

\boxheader{Constructor}
To create the underlying \texttt{solution} object, your solution constructor
should invoke the \texttt{solution(branching*)} constructor.  It should also
set the \texttt{value} member to reflect the solution's objective
value.  For the \texttt{\emph{myBranching}},
\texttt{\emph{myBranchSub}}, classes, for example, the standard
constructor pattern would be
\begin{codeblock}
\emph{mySolution}(\=\emph{myBranching}* global\_, \\
                  \>\textrm{[}, \textrm{\emph{other arguments}]}) : \\
~~~solution(global\_), \\
~~~$\cdots$ \\
\{ \\
~~~value = $\cdots$; \\
~~~\textrm{[\emph{Set internal representation of solution}]} \\
\}
\end{codeblock}
Note that in the invocation of \texttt{solution(global\_)}, the
\texttt{\emph{myBranching*}} pointer \texttt{global\_} is
automatically cast to a base class pointer of type
\texttt{branching*}. 

\boxheader{Destructor}
A destructor should be provided.  However, as noted above, it should not be
invoked directly through \texttt{delete}, but through the \texttt{solution}
class' \texttt{dispose()} method.

\method{const char* typeDescription() const} Returns a \texttt{const
  char*} describing the type of solution, for example
\texttt{"Knapsack solution"}.

\method{virtual void printContents(std::ostream\& s)} Write the
solution contents (for example the decision variable values) to the stream
\texttt{s}.  You do not need to write the objective value; it is
printed by another method.  The default implementation is a stub.

\method{virtual size\_type sequenceLength()}  Return the number of
elements in the sequence representation of the solution.  

\method{virtual double sequenceData()} Return the next element of the
sequence representation.

\vspace{2ex}

The following method is related to \texttt{sequenceData()} and
\texttt{sequenceLength()}, but may not have to be overridden:

\vspace{2ex}

\method{virtual void sequenceReset()} Cause the next call to
\texttt{sequenceData()} to return the first element in the sequence.
The default implementation is simply to set the data member
\texttt{sequenceCursor} (built into the \texttt{solution} class) to
zero.

\vspace{2ex}

Note that PEBBL will call \texttt{sequenceData()} at most
\texttt{sequenceLength()} times between calls to
\texttt{sequenceReset()}.  Typically, the default implementation of
\texttt{sequenceReset()} will suffice, and each call to
\texttt{sequenceData()} should return the
$\text{\texttt{sequenceCursor}}^\text{th}$ element of the sequence and
then increment \texttt{sequenceCursor}.

It is not absolutely necessary that a solution have a sequence
representation, and thus not absolutely required that you define
\texttt{sequenceLength()} and \texttt{sequenceData()}.  If you do not
define those methods, however, you must define two other methods:

\vspace{2ex}

\method{size\_type computeHashValue()} Return the hash value of the
solution.

\method{bool duplicateOf(solution\& other)} Return \texttt{true} if
\texttt{other} is an identical solution, and otherwise
\texttt{false}. 

\vspace{2ex}

Normally, it is recommended that you simply retain the default
implementations of these methods, and define application-specific
versions of \texttt{sequenceLength()}, \texttt{sequenceData()}, and
possibly \texttt{sequenceReset()}.  The default implementations of
\texttt{computeHashValue()} and \texttt{duplicateOf} are based on the
sequence representation.  Even if you have defined a sequence
representation, you may elect to override \texttt{computeHashValue()}
and/or \texttt{duplicateOf} to improve efficiency.


\subsubsection{The \texttt{foundSolution} method}
So far, the only means we have discussed for providing solutions to
PEBBL is the method \texttt{extractSolution} (see
Section~\ref{sec:bsubmethods}), which PEBBL calls between
\texttt{boundComputation} and \texttt{splitComputation} for
subproblems whose bounds are known to be exact.  Many branch-and-bound
algorithms encounter feasible solutions at other points in the search
process.  When this occurs, you may call the method
\texttt{branching::foundSolution(solution* sol)}, where \texttt{sol}
is a pointer to a \texttt{solution}-derived object.  You need not
first ascertain whether \texttt{*sol} is a better solution than the
incumbent, or, when enumerating multiple solutions, whether it is
eligible to enter the repository.  PEBBL will determine whether the
solution is useful, and otherwise delete it.  Note that calling
\texttt{foundSolution(sol)} cedes memory ownership of one reference to
\texttt{*sol} to PEBBL.  If the caller would like to continue using
\texttt{*sol} after calling \texttt{foundSolution(sol)}, the following apply:
\begin{itemize}
\item The caller should invoke \texttt{sol->incrementRefs()} before calling 
\texttt{foundSolution(sol)}
\item The caller should not alter the contents of \texttt{*sol}
solution after calling \texttt{foundSolution(sol)}, since doing so could cause
internal errors or unpredictable behavior in PEBBL
\item Once it no longer needs \texttt{*sol}, the caller should invoke
\texttt{sol->dispose()}.
\end{itemize}
If the caller does not have further need of \texttt{*sol} after calling
\texttt{foundSolution(sol)}, no action is necessary.

It is safe to call \texttt{foundSolution} at any point within the
branch-and-bound tree exploration, for example, from within the
methods \texttt{boundComputation}, \texttt{splitComputation} and
\texttt{makeChild}.  For example, from the \texttt{boundComputation}
method of the hypothetical \texttt{\emph{myBranchSub}} class, you
could invoke \texttt{foundSolution} via
\texttt{global()->foundSolution(sol)}.  You should not call
\texttt{foundSolution} at points before the branch-and-bound search
has begun (for example, in the \texttt{setupProblem} method or the
\texttt{preprocess} method described immediately below), or after the
search has concluded.  To provide a rough initial solution before the
tree exploration process has started, use the \texttt{initialGuess}
method described immediately below.


\subsubsection{Selected additional methods}
\label{sec:seradd}
We now present an inexhaustive list of additional methods that you may
wish to override for particular applications.

\method{void branching::preprocess()}
This method is intended for ``preprocessing'' of the problem prior to
commencing the branch-and-bound search; the default implementation is
an empty stub.  In principle, its functions
could be combined with \texttt{setupProblem($\cdots$)}, but it is
provided for clarity and to ease migration to the parallel layer.
%% Note that once you migrate your application
%% to the parallel layer, \texttt{preprocess()} is called \emph{after}
%% the problem instance is broadcast to all processors, and could
%% potentially be parallelized.

\method{solution* branching::initialGuess()} PEBBL calls this method
between preprocessing and the start of the branch and bound search.
Its intention is allow computation of a ``quick and dirty'' initial
guess at the solution.  In a knapsack problem, for example, this
routine could run a simple greedy heuristic.  Your implementation need
not be guaranteed to find a usable solution; if it does not, it should
just return a \texttt{NULL} pointer.  The default implementation is a
stub that simply returns \texttt{NULL}.

\method{bool branching::haveIncumbentHeuristic()} 
The default implementation of this method just returns \texttt{false}.
You should change it to return \texttt{true} if your application has a
way of trying to obtain a feasible solution from a non-terminal
\texttt{bounded} subproblem.  

\method{void branchSub::incumbentHeuristic()} 
This method applies only to the serial layer, and PEBBL will only call
it directly if the method \texttt{haveIncumbentHeuristic()} returns
\texttt{true}.  PEBBL calls it at two points, once a solution has
become \texttt{bounded}, and once it has become \texttt{separated}. 
The intention is to allow the heuristic calculation of feasible
solutions.  Whenever your implementation finds a feasible solution, it
should communicate it to PEBBL by calling \texttt{foundSolution}.  The
default implementation is a stub.

You may wish to check the subproblem state at the beginning of your
implementation of \texttt{incumbentHeuristic}.  For example, if you
wish your heuristic to be invoked only when subproblems become
\texttt{bounded}, and not when then become \texttt{separated}, you
could begin your implementation with
\begin{codeblock}
if (state != bounded) return;
\end{codeblock}
If you wish to construct heuristic solutions at other points besides
when subproblems become \texttt{bounded} or \texttt{separated}, you
can create your own heuristic method and invoke it at other times,
having it call \texttt{foundSolution} to send new solutions to PEBBL.

Note that the parallel layer treats incumbent heuristics somewhat
differently, and parallel-layer runs will not explicitly call
\texttt{incumbentHeuristic}; see Section~\ref{sec:parMethods} below.
 
\method{void branchSub::makeCurrentEffect()} PEBBL has the notion of a
``current subproblem'' -- the one presently being bounded or
separated.  The method \texttt{makeCurrentEffect()} provides a
``hook'' that is called whenever a subproblem is made ``current''.  In
a branch-and-cut algorithm, for example, this method could load the
subproblem's cuts into the linear programming solver.  Its default
implementation is a stub.

\method{void branchSub::noLongerCurrentEffect()}
PEBBL calls this ``hook'' whenever a subproblem ceases to be
``current''.  Again, its default implemetation is a stub.

\method{bool branchSub::forceStayCurrent()}
PEBL calls this method when it is considering replacing the current
subproblem with a different one.  Returning \texttt{true} indicates
that the current subproblem should be kept current if at all possible.
The default implementation always returns \texttt{false}, meaning that
PEBBL is free to ``unload'' the current subproblem, and replace it
with a different one.

\method{bool branching::checkParameters()} PEBBL calls this method to
make sure that its runtime parameters are set correctly and
consistently.  You may override this method to perform additional
checks specific to your application, and then call
\texttt{branching::checkParameters()} to make sure PEBBL's internal
checks are invoked.  You are most likely to need this routine if you
have defined your own application-specific runtime parameters; see the
section immediately below.

\subsubsection{Creating your own parameters}
\label{sec:ownparams} 
In addition to PEBBL's existing parameters, you may create your own
command-line parameters.  The simplest place to do so is in the
constructor of your \texttt{branching}-derived class.  Alternately,
you can define a underlying base class to hold your parameters: you
could create a class \texttt{\emph{myParams}} to hold your parameters,
and then derive \texttt{\emph{myBranching}} from both
\texttt{branching} and \texttt{\emph{myParams}}.  Parameter creation
statements should usually reside in constructors and have the general
form
% The following null command stops sublime text from  miscoloring 
% the rest of the file
\newcommand{\nothing}{}  
\begin{codeblock}
create\_categorized\_parameter(\="\textrm{\emph{commandLineName}}",\\
\>\textrm{\emph{internalName}},\\
\>"\nothing<\textrm{\emph{datatype}}>",\\
\>"\textrm{\emph{defaultValue}}",\\
\>"\textrm{\emph{text description}}",\\
\>"\textrm{\emph{category}}"\textrm{[},\textrm{]}\\
\>\textrm{[\emph{error check object}]})
\end{codeblock}
Here,
\begin{description}
\item[\textmd{\emph{commandLineName}}] is the name to be recognized on
  the command line.  For example, \texttt{fooBar} will cause the
  option \texttt{--fooBar=\textrm{\emph{value}}} to be recognized on the
  command line.  
\item[\textmd{\emph{internalName}}] is a reference to a class data
  member in which to store the value of the parameter; usually,
  \emph{internalName} and \emph{commandLineName} should be identical
  for clarity.
\item[\textmd{\emph{datatype}}] is the C++ datatype of
  \emph{internalName}.  Common choices are \texttt{bool},
  \texttt{int}, \texttt{double}, and \texttt{string}.
\item[\textmd{\emph{defaultValue}}] is a text representation of 
  the default value for the
  parameter.  It should match the initial value stored in the data
  member \emph{internalName}.  Note that the specification of
  \emph{defaultValue} does not automatically initialize the value of
  \emph{internalName}; its initial value should be set by the class
  constructor and \texttt{reset} methods.
\item[\textmd{\emph{text description}}] is a parameter description
  printed in response to \texttt{--help}.  For descriptions longer
  than one line, embed newline-tab character sequences
  (``\texttt{$\backslash$n$\backslash$t}'') in this text.
\item[\textmd{\emph{category}}] The category header to be used when
  printing the parameter description in response to \texttt{--help}.
  You may invent new categories.
\item[\textmd{\emph{error check object}}] is optional and 
  specifies simple constraints
  on the value of the parameter.  For example,
  \begin{codeblock}
  ParameterNonnegative<int>()
  \end{codeblock}
  specifies that an integer parameter must be nonnegative, while
  \begin{codeblock}
  ParameterBounds<double>(0.0,1.0)
  \end{codeblock}
  specifies that a \texttt{double}-datatype parameter must have values
  between 0 and 1, inclusive.  Other common forms for error-check objects are
  \begin{codeblock}
  ~~~~\=ParameterLowerBound<\textit{datatype}>(\textit{minimum
      value}) \\
  \textrm{or}\>
  ParameterUpperBound<\textit{datatype}>(\textit{maximum
      value})
  \end{codeblock}
  \noindent For more details, refer to the UTILIB \texttt{Parameter}
  and related class documentation.
\end{description}

For error checks that are more elaborate, or involve interactions
between multiple parameters, you may specify parameter error checks by
overloading the \texttt{checkParameters} boolean method in the
\texttt{branching} class, for example with
\texttt{\textit{myBranching}::checkParameters}; if this routine
detects an error, it should print a diagnostic message and return
\texttt{false}.  If it does not, it should execute \texttt{return
  branching::checkParameters()} to invoke the standard checks on
PEBBL's built-in parameters.

Note that for parameters of type \texttt{bool}, you may omit
\texttt{=}\emph{value} on the command line, in which case the value is
set to \texttt{true}.  That is, \texttt{--fooBar} is
equivalent to \texttt{--fooBar=true}


\subsubsection{Serial main programs: drivers}
\label{sec:serialdriver}
PEBBL is structured as a callable library, and can thus be
incorporated directly into other C++ applications.  If you want your
PEBBL application to function as a simple ``solver'' executable that
simply reads in a problem instance data file and solves it, you need
to create a ``driver'' main program.  A simple template function called
\texttt{driver} is available for this purpose.  Using this
template, you can construct the driver for the hypothetical example
application \texttt{\emph{myBranching}} as follows:
\begin{codeblock}
\textrm{\emph{Header file}} \#include\textrm{\emph{s, including}} 
\emph{myBranching} \textrm{\emph{class and}} <pebbl/branching.h> \\
\\
int \=main(int argc, char** argv)\\
\{\\
  \> return driver<\emph{myBranching}>(argc,argv);\\
\}
\end{codeblock}
Invoking \texttt{driver<\emph{myBranching}>(argc,argv)} 
is equivalent to the code shown in Figure~\ref{fig:serialdriver}.
Note that the main PEBBL header file \texttt{<pebbl/branching.h>}
automatically includes the necessary UTILIB header files to define the
method \texttt{InitializeTiming()}.
\begin{figure}[tbp]
\begin{center}
\fbox{
\begin{minipage}{0.65\textwidth}
\vspace{1ex}
\begin{codeblock}
InitializeTiming(); \\
  \emph{myBranching} instance; \\
  boo\=l f\=lag = instance.setup(argc,argv); \\
  if (flag) \\
\>    \{ \\
\>\>      instance.reset(); \\
\>\>      instance.solve(); \\
\>    \}\\
  return !flag;
\end{codeblock}
\vspace{1ex}
\end{minipage}
}
\end{center}
\vspace{-2ex}
\caption{Code equivalent to the serial
  \texttt{driver<\emph{myBranching}>(argc,argv)} template.}
\label{fig:serialdriver}
\end{figure}
Note also that \texttt{setup($\cdots$)}, \texttt{reset()}, and
\texttt{solve()} are all methods of the \texttt{branching} class that
you typically do not override.  Calling \texttt{setup(argc,argv)}
parses the command line, extracting and processing all command line
arguments recognizable as PEBBL parameters, including any 
application-specific parameters created as described in 
Section~\ref{sec:ownparams}.  It then calls your
implementation of \texttt{setupProblem}, with \texttt{argc} and
\texttt{argv} adjusted so that parameter-setting arguments are
removed; \texttt{setup}'s returning \texttt{true} indicates both that
all command line parameters were processed correctly, and that your
\texttt{setupProblem} implementation returned \texttt{true},
indicating it was successful.  The method \texttt{solve()} invokes the
branch-and-bound search engine, and prints subproblem count and timing
statistics upon termination.  It also uses your
\texttt{solution}-derived class' \texttt{printContents} implementation
to write the final solution(s) to a file whose name is derived from
the first command line argument not recognizable as a parameter
setting, typically by appending \texttt{sol.txt}; if no such parameter
can be found, the file is called \texttt{sol.txt}).  If you instead
wish to specify your own choice for the solution file, use the
\texttt{--output=}\emph{filename} command-line option.  Here,
\emph{filename} may contain path information, and thus need not be in
the current directory.

The \texttt{setup} method manages parsing of the command line
including option specifications and a problem instance data file name.
Supposing your driver were called \texttt{\emph{myDriver}}, the
command line \texttt{\emph{myDriver} \emph{datafile}} 
would call \texttt{\emph{myBranching}::setupProblem(argc,argv)} with
\texttt{argv[1]} containing the null-terminated string
\texttt{\emph{datafile}} --- with a typical implementation of
\texttt{setupProblem}, this would cause your application to read in
the problem instance description in \texttt{\emph{datafile}}.
In parsing the command line, \texttt{setup} will also recognize all PEBBL
command-line parameters (see Section~\ref{sec:param} for a partial
catalog) along with any parameters defined for
\texttt{\emph{myBranching}}, as described in
Section~\ref{sec:ownparams}.  For example, in the command line
\begin{codeblock}
\emph{myDriver} --relTolerance=0.05 --earlyOutputMinutes=2 \emph{datafile}
\end{codeblock}
the PEBBL parameters \texttt{relTolerance} and
\texttt{earlyOutputMinutes} would be set to $0.05$ and $2$,
respectively.  In this example,
\texttt{\emph{myBranching}::setupProblem(argc,argv)} would still be
called with \texttt{argv[1]} containing the null-terminated string
\texttt{\emph{datafile}}, presumably reading the problem instance in
\texttt{\emph{datafile}}. 
The \texttt{setup} method also automatically recognizes several
special command-line arguments:
\begin{description}
\item[\texttt{--help}] Causes \texttt{setup} to print a usage line
followed by a description of all available parameters, and then return
\texttt{false} (so that the driver above would not attempt to solve a
problem).  If you want to control the form of the usage line, override
the method \texttt{void branching::write\_usage\_info(char*
  progName,std::ostream\& os)}.  When PEBBL calls this method, the
argument \texttt{progName} is set equal to \texttt{argv[0]}.
\item[\texttt{--version}] must be the first argument if present.  It
  causes \texttt{setup} to print the value of the \texttt{static}
  \texttt{std::string} data member \texttt{branching::version\_info} and return
  \texttt{false} (so that the driver above would then immediately
  exit).  You may alter the contents of \texttt{version\_info}, for
  example, in the constructor for your \texttt{branching}-derived
  class.
\item[\texttt{--param-file=}\textmd{\emph{file}}] allows multiple parameter
  settings to be read from the file \emph{file} --- see the UTILIB
  Parameter class documentation for a description of the file format.
\end{description}

The \texttt{reset()} method invoked by the driver template performs
all necessary initializations prior to the solution process, which is
performed by \texttt{solve()}.  The \texttt{solve()} method in turn
invokes several lower-level methods, principally:
\begin{description}
\item{\texttt{search()}} to perform the actual branch-and-bound
  search.  
\item{\texttt{printAllStatistics()}} to print information about the
  number of search nodes and run time.
\item{\texttt{solutionToFile()}} to write the solution file.
\end{description}

\subsubsection{``Quiet'' operation}
If PEBBL is embedded within another application, it may be desirable to
have it run ``quietly'', without any configuration, status, or
performance-statistics printouts.  In this case, you should use the
\texttt{search()} routine instead of \texttt{solve()} and make sure
that solution status printouts are disabled by setting both
\texttt{branching::statusPrintCount} and \texttt{statusPrintSeconds}
to zero (see Section~\ref{sec:outputparams} for discussion of the
corresponding runtime parameters).

\subsubsection{Retrieving solutions to C++ code}
\label{sec:serialgetsol}
Note that PEBBL applications can be invoked in ways other than the
simple driver above; for example, they could be embedded in more
complicated C++ programs, in which case you would probably use the
\texttt{search()} method to perform the branch-and-bound search, and
avoid command-line-oriented methods like \texttt{driver},
\texttt{setupProblem}, and \texttt{solve()}.  

Various approaches are available to retrieve solutions computed by
running the \texttt{search()} method; note that all the methods described in
this subsection are members of the \texttt{branching} class.  If you
are not enumerating multiple solutions, you should simply use the method
\texttt{getSolution()}, which returns a \texttt{solution*} pointer to
the incumbent solution.  This method returns \texttt{NULL} if no
solution was found.  Within your application, the \texttt{solution*}
pointer returned by \texttt{getSolution()} will most likely need to be
cast ``up'' the inheritance hierarchy so that its contents may be
accessed, preferably using the C++ \texttt{dynamic\_cast} operator.

For retrieving multiple solutions from a run enumerating multiple
solutions, the simplest technique is to define an object
\texttt{\emph{solArray}} of type \texttt{BasicArray<solution*>} and
then call the method \texttt{getAllSolutions(\emph{solArray})}.  This
method will redimension \texttt{\emph{solArray}} to the size of the
repository and fill it with \texttt{solution*} pointers to the
repository members, in objective-value order, starting with the best
value.

For compatibility with the parallel layer, a second, iterator-like
technique is also possible.  In this approach, one first calls
\texttt{startRepositoryScan()}, which returns the size $n$ of the
repository.  One may then call \texttt{nextRepositoryMember()} up to
$n$ times: each time, it will return a different member of the
repository.  Again, solutions are returned in objective value order,
starting with the best values.

For any solution pointer \emph{\texttt{sol}} returned by \texttt{getSolution},
\texttt{getAllSolutions}, or \texttt{next\-Repos\-itory\-Member}, the application
should call \texttt{\emph{sol}->dispose()} when it no longer needs the
corresponding solution object.  As discussed in
Section~\ref{sec:solclassdetails}, one should never directly use
\texttt{delete} on \texttt{solution}-derived objects, because they are
reference-counted.

\subsubsection{Debugging outputs}
In the process of debugging a PEBBL application, it may be necessary
to print diagnostic outputs.  Through the companion UTILIB package
that it automatically includes, PEBBL provides special mechanisms for
such output.  The most frequently used is the \texttt{DEBUGPR} macro,
which takes the form
\begin{codeblock}
DEBUGPR(\textit{level},\textit{action});
\end{codeblock}
Inside a class derived from \texttt{branching}, such a statement
indicates that the code given by \texttt{\textit{action}} be executed
if the PEBBL debug output level specified by the run-time parameter
\texttt{debug} in Section~\ref{sec:debugparams} has a value of
\texttt{\textit{level}} or higher.  Typically,
\texttt{\textit{action}} should involve writing text to the
specialized stream \texttt{ucout}.  For example, the statement
\begin{codeblock}
DEBUGPR(10,ucout << "Value of n is " << n << endl);
\end{codeblock}
will print the value of the variable \texttt{n} whenever the debug output
level is at 10 or higher.  While it makes not difference in the serial
layer, using the special stream \texttt{ucout} is preferable to using
the standard stream \texttt{cout} because \texttt{ucout} has special
features that make it more convenient in a parallel setting.  In
particular, when used in parallel, each line of text sent to
\texttt{ucout} will be prefixed by a tag indicating which processor
wrote the information.

A slightly different form of debug printing is used in subproblem
classes derived from \texttt{branchSub}, as opposed to problem classes
derived from \texttt{branching}.  In this setting, the diagnostic
output above would instead be expressed as
\begin{codeblock}
DEBUGPRX(10,global(),ucout << "Value of n is " << n << endl);
\end{codeblock}
The reason for this difference is that, for efficiency reasons,
\texttt{branchSub} does not derive from the same UTILIB
\texttt{CommonIO} utility class that \texttt{branching} does.  

The \texttt{CommonIO} class from UTILIB makes several other variations
on the \texttt{DEBUGPR} available for specialized situations.  Consult
the UTILIB documentation for more information.

An advantage of using the \texttt{DEBUGPR} and related facilities
instead of \emph{ad hoc} print statements is that the diagnostic
output may be left in the code indefinitely and triggered whenever
needed by using the \texttt{debug} runtime parameter.  For example,
running a PEBBL application with the command-line option
\texttt{--debug=10} would set the debugging level to 10.  

\subsection{Defining a parallel application}
\label{sec:parMethods}
Now suppose that your serial layer applications runs acceptably, and
you wish to parallelize it.  If you want to use parallelism, be sure
that you have configured PEBBL with MPI; see Section~\ref{sec:compiling}, and
especially Section~\ref{sec:configoptions}.  In particular, you should
run \texttt{configure} with an option such as
\texttt{--with-mpi-compilers=yes} or
\texttt{--with-mpi-compilers=}\emph{pathname}.

When Acro is compiled with MPI, the preprocessor symbol
\texttt{ACRO\_HAVE\_MPI} is defined; otherwise it is not defined.
When \texttt{ACRO\_HAVE\_MPI} is undefined, only the serial portions
of PEBBL are compiled; the parallel portions are omitted or become
stubs.  You may wish to follow this same pattern within your own
application source code.

Figure~\ref{fig:code2} outlines the
recommended inheritance and pointer pattern for creating parallel
layer classes from a serial application.  Here, the serial layer
classes are \texttt{\emph{myBranching}} and
\texttt{\emph{myBranchSub}}, and the corresponding 
respective parallel layer classes
are \texttt{\emph{myParBranching}} and \texttt{\emph{myParSub}}.  Note
that the header file \texttt{<pebbl/parBranching.h>} defines all the
classes in both the serial and parallel layers.  

\begin{figure}[tbp]
\begin{center}
\fbox{
\begin{minipage}{0.85\textwidth}
\vspace{1ex}
\begin{codeblock}
\#include <pebbl/parBranching.h> \\
\\
using namespace pebbl; \\
\\
class \emph{myParSub};  ~~~~// Forward declaration \\
\\
cla\=ss \emph{myParBranching} : \\
\>virtual public parallelBranching, \\
\>virtual public \emph{myBranching} \\
\{\\
~public: \\
\>$\quad\vdots$ \\
\>\textit{myParBranching\textrm{ methods and data here}} \\
\>$\quad\vdots$ \\
\}; \\
\\
class \emph{myParSub} : \\
\> virtual public parallelBranchSub, \\
\> virtual public \emph{myBranchSub} \\
\{\\
~protected:\\
\>  // A pointer to the global parallel branching object\\
\>  \emph{myParBranching*} globalPtr;\\
~public:\\
\>  // Return a pointer to the global branching object\\
\>  \emph{myParBranching}* global() const \{ return globalPtr; \}\\
\\
\>  // Return a pointer to the parallel global base class object\\
\>  parallelBranching* pGlobal() const \{ return global(); \}\\
\>$\quad\vdots$ \\
\>\textit{myParSub\textrm{ methods and data here}} \\
\>$\quad\vdots$ \\
\};
\end{codeblock}
\vspace{1ex}
\end{minipage}
}
\end{center}
\vspace{-2ex}
\caption{Standard code pattern for creating a parallel PEBBL application.}
\label{fig:code2}
\end{figure}
Note that PEBBL uses \texttt{parallelBranchSub::pGlobal()} 
to find the
\texttt{parBranching} object associated with a subproblem object. The
method \texttt{global()} in Figure~\ref{fig:code2} would be for your
own use and might not be necessary if your
\texttt{parBranching}-derived class does not encapsulate significantly
more data than your \texttt{branching}-derived class.

Your \texttt{parBranching}-derived class may also contain additional
parameters not present in your \texttt{branching}-derived class.  The
method for the creating such additional parameters is identical to
that of
Section~\ref{sec:ownparams}, but the corresponding data members should
be in
your \texttt{parBranching}-derived class, for example
\texttt{\emph{myParBranching}}, and the necessary matching calls to
\texttt{create\_categorized\_parameter} should be in that class'
constructor. 


\subsubsection{Methods you should create ---
  \texttt{parallelBranching}-derived class}
\boxheader{Constructor}
A constructor with no arguments is advised; it will automatically call
the no-argument constructor for your serial application class, such as
\texttt{\emph{myBranching}}. 
\begin{codeblock}
\emph{myParBranching}() \{ $\cdots$ \}; \\
\end{codeblock}

\boxheader{Destructor} 
You should also have a destructor:
\begin{codeblock}
$\sim$\emph{myParBranching}() \{ $\cdots$ \}; \\
\end{codeblock}

\method{void reset(bool VBflag = true)}
Perform all needed initializations subsequent to reading in the
problem instance.  This process consists of the following steps (not
necessarily in the given order):
\begin{itemize}
\item Invoke the underlying serial problem-instance class
  \texttt{reset} method, for example by
  \texttt{\emph{myBranching}::reset}.
\item Register at least one reference solution; see
  Section~\ref{sec:parsol}.
\item Invoke the \texttt{parBranching::reset} method.
\item Perform any additional application-specific initializations that
  are needed only by parallel runs.
\end{itemize}
The \texttt{VBflag} argument has the same purpose as the
\texttt{resetVB} argument in serial \texttt{reset} methods.  See
Section~\ref{sec:parsol} below for an example implementation of the
parallel \texttt{reset} method.

\method{parallelBranchSub* blankParallelSub()} This method is similar to
\texttt{blankSub}, but should return a parallel subproblem with
correctly initialized global pointers, as in the following code; see
Section~\ref{sec:parsubmethods} below for a possible implementation of
\texttt{\emph{myParSub}::setGlobalInfo}.
\begin{codeblock}
par\=allelBranchSub* \emph{myBranching}::blankParallelSub() \\
\{ \\
\>   \emph{myParSub}* newSP = new \emph{myParSub}; \\
\>    newSP->setGlobalInfo(this);\\
\>    return newSP; \\
\};
\end{codeblock}

\method{void pack(utilib::PackBuffer\& outBuffer)} PEBBL uses this
method when broadcasting the problem description.  It should write all
the the information describing a problem instance (\emph{i.e.}
everything read by \texttt{setupProblem($\cdots$)}) into the UTILIB
\texttt{PackBuffer} supplied in the argument \texttt{outBuffer}.
Writing to a \texttt{PackBuffer} is very similar to writing to an
unformatted stream: for most native C++ datatypes, the \texttt{<<}
operator, that is, ``\texttt{outBuffer <<} \emph{data}'' will write
scalar data.  Entire UTILIB arrays, for example of datatype
\texttt{BasicArray<\textit{T}>}, 
\texttt{IntVector} or \texttt{DoubleVector}, may also be written with
\texttt{<<}, as can STL \texttt{set}s or 
\texttt{vector}s of standard C++ base types.
See the UTILIB documentation for the details of \texttt{PackBuffer}s.

\method{void unpack(utilib::UnPackBuffer\& inBuffer)} 
Again, PEBBL
uses this method when broadcasting the problem description.  Its job
is to read from
\texttt{inBuffer} the information written by \texttt{pack}.  Most native C++
datatypes, along with many UTILIB-supplied containers and STL
\texttt{set}s or
\texttt{vector}s of simple types, may be read via
the the \texttt{>>} operator, applied in exactly the same order as you
used \texttt{<<} in \texttt{pack}.  For example, data written by
\begin{codeblock}
outBuffer << time << money;
\end{codeblock}
in \texttt{pack} might be read via
\begin{codeblock}
inBuffer >> time >> money;
\end{codeblock}
in \texttt{unpack}.  
%% The \texttt{>>} operator can also read entire UTILIB arrays.

\method{int spPackSize()} This method should return an upper bound on
the number of bytes needed to pack all the information to describe
\emph{a subproblem} --- that is, the maximum amount of space needed by
the method \texttt{\emph{myParSub}::pack($\cdots$)} described below.
It does \emph{not} refer to the amount of space needed by
\texttt{\emph{myParBranching}::pack}, which PEBBL is able to detect
automatically. In \texttt{PackBuffer}s and \texttt{UnPackBuffer}s,
most C++ native datatypes require the same amount of space as their
machine representations: \emph{i.e} a data member of type
\texttt{\emph{x}} requires \texttt{sizeof(\emph{x})} bytes.  You
should use \texttt{sizeof($\cdot$)} in your implementation to make
sure it is portable between 32- and 64-bit architectures and varying
compilers.  For standard container datatypes such as 
UTILIB arrays, STL \texttt{set}s, or STL
\texttt{vector}s of type \texttt{\emph{T}}, the space required is a single
\texttt{sizeof(size\_t)} to hold the size of the container, plus
$m\cdot\text{\texttt{sizeof(\emph{T})}}$, where $m$ is the maximum
container size possible for the given problem instance.

The \texttt{spPackSize()} method is
called after the problem has been read and broadcast, so all
information set by \texttt{\emph{myBranching}::setupProblem} and
\texttt{\emph{myParBranching}::unpack} should be available to it in all
processors.

Instead of direct calculation, one possible approach to implementing
\texttt{spPackSize()} is to create a ``dummy'' subproblem of the largest
possible size, invoke your \texttt{\emph{myParSub}::pack($\cdots$)}
method to write it to a dummy buffer \texttt{\emph{dumBuf}}, and then measure
the size of that buffer with \texttt{\emph{dumBuf}.size()}.

~

\subsubsection{Methods you should create ---
  \texttt{parallelBranchSub}-derived class}
\label{sec:parsubmethods}
\boxheader{Constructor}
Again, you need an empty-argument constructor, for example:
\begin{codeblock}
\emph{myParSub}() \{  $\cdots$ \}; \\
\end{codeblock}

\boxheader{Destructor} 
You also need a destructor:
\begin{codeblock}
$\sim$\emph{myParSub}() \{ $\cdots$ \}; \\
\end{codeblock}

\method{parallelBranching* pGlobal() const}
This method is similar to \texttt{bGlobal}, but returns a pointer of
type \texttt{parallelBranching*}, implementing the third dashed arrow
in Figure~\ref{fig:parinherit}.  Given \texttt{globalPtr} as defined
in Figure~\ref{fig:code2}, it could be implemented via
\begin{codeblock}
myParBranching* global() const \{ return globalPtr; \} \\
parallelBranching* pGlobal() const \{ return global(); \}
\end{codeblock}

\method{void setGlobalInfo(\emph{myParBranching*} global\_)} 
As in the
serial, case, the purpose of this routine should be to ``bind'' a
particular subproblem object to the problem instance description embodied in
the object \texttt{global\_}.  It should also make sure that the
corresponding serial-layer binding is also performed.  For example:
\begin{codeblock}
voi\=d setGlobalInfo(\emph{myParBranching*} global\_) \\
\{ \\
\> globalPtr = global\_; \\
\> \emph{myBranching}::setGlobalInfo(global\_);
  // Sets serial layer pointer etc. \\
\> $\vdots$\\
\}
\end{codeblock}

\method{void pack(utilib::PackBuffer\& outBuffer)}
This method should pack the application-specific portion of the 
description of the subproblem into the \texttt{PackBuffer} object
\texttt{outBuffer}, typically using the \texttt{<<} operator.  For
more details about writing to \texttt{UnPackBuffer} objects, see the
description of \texttt{branching::unpack} above.

\method{void unpack(utilib::UnPackBuffer\& inBuffer)}
This method should unpack the application-specific portion of the
description of the subproblem from the \texttt{UnPackBuffer} object
\texttt{inBuffer}, typically using the \texttt{>>} operator.  For
more details about reading from \texttt{PackBuffer} objects, see the
description of \texttt{branching::pack} above.

\method{virtual parallelBranchSub* makeParallelChild(int whichChild)}
This method is similar to \texttt{makeChild}, but returns a
\texttt{parallelBranchSub*}.  It should create the
\texttt{whichChild}'th child of the present subproblem, counting from
$0$ to $k-1$, where $k$ is the number of children.  PEBBL only calls
this method for subproblems in the \texttt{separated} state.

Note: \texttt{p->makeParallelChild(whichChild)} should \emph{not} have
the side-effect of changing the bound of \texttt{*p}, or memory errors
may occur.  This restriction may be removed in subsequent PEBBL
releases.


\subsubsection{Standard disambiguations}
While the ``diamond'' inheritance pattern shown in
Figure~\ref{fig:layers} is powerful, it may also lead
to some ambiguities.  When an inherited method is defined in several
different base classes, the C++ compiler may be unsure which
implementation to use.  If such ambiguity occurs when compiling a
parallel PEBBL application, the general rule is to use the
implementation in either \texttt{parallelBranching} or
\texttt{parallelBranchSub}, which will in turn automatically call the
appropriate serial layer routine.  In a class such as
\texttt{\textit{myParBranching}}, for example, it may be necessary to
define
\begin{codeblock}
boo\=l s\=etup(int\& argc,char**\& argv) \\
\>    \{  \\
\>\>      return parallelBranching::setup(argc,argv); \\
 \>   \}
\end{codeblock}
and similarly for a few other routines.


\subsubsection{Parallel solution management and parallel \texttt{reset}}
\label{sec:parsol}
PEBBL does not need distinct serial and parallel versions of the class
representing problem solutions; the same \texttt{solution} class
fulfills this role in both serial and parallel settings.  To function
in a parallel settings, however, user-defined
\texttt{solution}-derived classes require a few additional methods
beyond those mentioned in Section~\ref{sec:solclassdetails}.
Specifically, you should define implementations of the following
\texttt{virtual} methods:

\method{solution* blankClone()} This method should return 
a new ``blank'' version of the
current solution, that is, a (derived) solution object 
of the same type and dimensions.  
The blank copy should contain all information obtained or
copied from your \texttt{branching}-derived class (for example, from
the \texttt{\emph{myBranching}} class).  This information should
include any pointers to the problem instance class, and the correct
setting of \texttt{sense}.  For all such information embedded in the
\texttt{solution} base class, the constructor
\texttt{solution(solution*)} can be useful; a standard implementation
pattern might be
\begin{codeblock}
solution* blankClone() \{ return new \emph{mySolution}(this); \}; \\
\\
\emph{mySolution}(\emph{mySolution}* toCopy) : \\
~~~\=solution(toCopy), \\
\> $\cdots$\textrm{\emph{other initializations}}~$\cdots$ \\
\{ ~ $\cdots$ ~ \};
\end{codeblock}
It is not necessary for the object returned by \texttt{blankClone()}
to contain any valid information about a particular solution to your
problem instance.

\method{void packContents(PackBuffer\& outBuf)} Pack the
application-specific information in the current solution into the
\texttt{PackBuffer} object \texttt{outBuf}, typically using the
\texttt{>>} operator.  For more details about writing to
\texttt{PackBuffer} objects, see the description of
\texttt{branching::pack} above.

\method{void unpackContents(UnPackBuffer\& inBuf)}
Unpack application-specific solution information from the
\texttt{UnPackBuffer} object \texttt{inBuf}, typically using the
\texttt{<<} operator.  The order of unpacking should be identical to
the order of packing in \texttt{packContents}.    For
more details about reading from \texttt{UnPackBuffer} objects, see the
description of \texttt{branching::unpack} above.

\method{int maxContentsBufSize()} This method should return an upper
bound on the maximum number of bytes
that calls to \texttt{packContents(outBuf)} might write to the buffer
\texttt{outBuf} for a solution to the current problem instance.  For more
information about how to calculate this buffer size, see the
discussion of the \texttt{parallelBranching::spPackSize()} method
above. 

\vspace{2ex}

PEBBL has its own methods for packing and unpacking information
contained in the \texttt{solution} base class; \texttt{packContents}
\texttt{unpackContents}, and \texttt{maxContentsBufSize()} do not need
to take such information into account.

The interprocessor communication mechanism for solutions requires that
you ``register'' a pointer to a ``reference'' object for each
\texttt{solution}-derived class used by your application.  It is
recommended that you perform this operation in your parallel
\texttt{reset} method, for example,
\texttt{\emph{myParBranching}::reset}.  Reference solutions have three
purposes:
\begin{itemize}
\item To calculate the maximum buffer size needed to receive a
  solution object: for each reference solution, PEBBL will call the
  method \texttt{maxContentsBufferSize}, and incorporate the results
  into its buffer sizing calculations.
\item To create new solutions from messages: when a processor receives
  a message containing a solution, PEBBL calls the \texttt{blankClone}
  method for one of the reference solutions producing a new object
  $s$.  PEBBL then reads the solution message into $s$ by invoking
  $s$\texttt{.unpack}.
\item To keep track of multiple solutions types: each kind of solution
  is assigned a \texttt{typeId} value, which is stored in the
  reference solution.  When a processor receives a message containing a
  solution, this \texttt{typeId} value allows PEBBL to determine what
  kind of solution object it encodes.
\end{itemize}
Usually, a given application uses only one kind of solution, in which
case the registration process is simple.  You need only create a
single solution object and pass it to the method
\texttt{parallelBranching::registerFirstSolution(solution*)}.  This
object should be configured so that its \texttt{maxContentsBufSize()}
method will return the correct value for the current problem instance.
For example, if your application uses the \texttt{arraySolution<int>}
solution representation and the maximum length of a solution is $n$,
your \texttt{\emph{myParBranching}::reset} would contain
\begin{codeblock}
registerFirstSolution(new arraySolution<int>($n$,this));
\end{codeblock}
The constructor \texttt{arraySolution<int>($n$,this)} creates an array
solution object of length $n$ (of indeterminant contents); it can also
accept optional arguments for a type name (\emph{e.g.} \texttt{"My
  application solution"}), and an array of variable names.
Constructing the reference solution with the maximum possible length
$n$ ensures that PEBBL's internal message buffers will be long
enough for any message encountered.  

If you use an application-specific \texttt{solution}-derived class,
make sure your reference solution's \texttt{maxContentsBufferSize}
method will return the maximum necessary space necessary to pack the
solution's contents for the current problem instance, 
and its \texttt{blankClone} method will create
properly formed solutions ready to have their \texttt{unpack} methods
called on the results of any message \texttt{pack} operation.  Assuming the
\texttt{\emph{mySolution}(\emph{myBranching}*)} constructor
of Section~\ref{sec:solclassdetails} behaves in the manner, a typical
form for the parallel \texttt{reset} method is
\begin{codeblock}
void \emph{myParBranching}::reset(bool vbFlag=true) \\
\{ \\
~~~\= \emph{myBranching}::reset();\\
   \> registerFirstSolution(new \emph{mySolution}(this));\\
   \> parallelBranching::reset(); \\
   \> ~~~\=$\vdots$ \\
   \> \textrm{\emph{Any additional application-specific parallel-only
                    initializations}} \\
   \>\> $\vdots$ \\
\}
\end{codeblock}

If your application uses more than one type of solution
representation, you must register a reference solution for each type.
You should register the first with
\texttt{registerFirstSolution}, and the rest with
\texttt{registerSolution}, which also takes a single
\texttt{solution*} argument.  Equivalently, you can call
\texttt{clearRegisteredSolutions()}, and then invoke
\texttt{registerSolution} for each reference solution.  Each solution
type will be assigned an identifier of C++ type \texttt{size\_type},
which is returned from the \texttt{registerFirstSolution} or
\texttt{registerSolution} call, and is also stored in the corresponding
reference solution's \texttt{typeId} member.  Whenever you create a
new solution, make sure its \texttt{typeId} member is set to the
identifier corresponding to its type (this should happen
automatically, for instance, if you create all new solutions by
applying \texttt{blankClone} to the corresponding reference solution).

Note that calling \texttt{registerFirstSolution} or
\texttt{registerSolution} cedes ownership of one reference to the argument to
PEBBL.  Therefore it is not typically necessary to call the \texttt{dispose()}
method of the registered solutions.  PEBBL will do this automatically upon
descruction or reset of a \texttt{branching}-derived object, or a call to
\texttt{clearRegisteredSolutions()}.

In the parallel setting, you still make PEBBL aware of new solutions
using the same \texttt{initialGuess}, \texttt{extractSolution}, and
\texttt{foundSolution} methods as in serial applications.  However,
these methods have more complex behavior in the parallel layer and
may now trigger various forms of interprocessor communication.  If you
are not using enumeration, the communication is typically a simple broadcast of
a possible new incumbent value.  If enumeration is active, then
new solutions may be ``hashed'' to other processor to check for
duplication and for incorporation into the global solution repository.
The \texttt{foundSolution} method also has special behavior during the
ramp-up phase, see Section~\ref{sec:rampupdetails} below.


\subsubsection{Incumbent heuristics in parallel}
PEBBL's parallel layer provides facilities allowing careful control
over how much CPU time per processor is spent on incumbent
heuristics.  In the parallel setting, there are two possible levels of
incumbent heuristic: a ``quick'' incumbent heuristic that is run for
each bounded or separated 
subproblem, much as in the serial layer, and a separate
incumbent thread whose CPU usage is controlled by the thread
scheduler.  We now consider the methods that implement this
functionality.  As in the serial layer, incumbent heuristics
communicate solutions to PEBBL by calling \texttt{foundSolution}.

\method{void parallelBranchSub::quickIncumbentHeuristic()} The
heuristic to be run for every \texttt{bounded} or \texttt{separated}
subproblem, similarly to the method 
\texttt{branching::incumbentHeuristic} in the
serial layer.  The default implementation is a stub.  You need not
attempt to truly run your heuristic every time
\texttt{quickIncumbentHeuristic()} is called.  For example, your
implementation could immediately return if the subproblem has the
wrong state, or does not look particularly ``attractive''.  If you do
run your heuristic and find an improved incumbent, you should call
\texttt{updateIncumbent()} or perform a similar sequence of
operations.  The process of calling \texttt{quickIncumbentHeuristic()}
is separate from the heuristic thread and does not require existence
of the heuristic thread.  If you wish the incumbent heuristic
calculations to be identical in the serial and parallel layers, make
sure \texttt{hasParallelIncumbentHeuristic()} (see immediately below)
returns \texttt{false}, and implement \texttt{quickIncumbentHeuristic}
as follows:
\begin{codeblock}
void\=~\textit{myParSub}::quickIncumbentHeuristic() \\
\> \{ \textit{myBranchSub}::incumbentHeuristic(); \} ;
\end{codeblock}

\method{bool parallelBranching::hasParallelIncumbentHeuristic()}
Returns \texttt{false} by default.  Return \texttt{true} if your
implementation has the capability to run a heuristic thread.  PEBBL
will create a heuristic thread if this method returns \texttt{true} and
the parameter \texttt{useIncumbentThread} is \texttt{true}.

\method{void parallelBranchSub::feedToIncumbentThread()}
This method is a ``hook'' called for each bounded subproblem.  It is
intended to examine a subproblem, and if it seems
sufficiently attractive, copy some representation of it to the data
structures used by the routine \texttt{parallelIncumbentHeuristic}.
In creating these data structures, you may want to make them able to
store representations of more than one subproblem.

\method{void parallelIncumbentHeuristic(double* controlParam)}
This is the method invoked by the incumbent thread.  The argument
\texttt{controlParam} is set by the scheduler to try to control the
amount of CPU time each call uses.  If you wish, you may ignore the
value of this argument, and simply set \texttt{*controlParam = 1} upon
exit. If you wish to be more responsive to the scheduler, try to do an
amount of work roughly proportional (in some sense of your own
choosing) to \texttt{*controlParam}, and then set
\texttt{*controlParam} equal to the amount of work performed.  

\method{ThreadObj::ThreadState incumbentHeuristicState()} 
This method indicates
whether the incumbent thread is ready to run.  The default
implementation is to always return \texttt{ThreadObj::ThreadBlocked},
indicating the thread is unable to run.  As soon as your thread has
some data upon which to operate, you should return
\texttt{ThreadObj::ThreadReady} instead.  

\method{double incumbentThreadBias()}
This method indicates the importance of running the incumbent
heuristic relative to the regular branch-and-bound worker process. The
default implementation uses a formula involving various standard parameters and
the current relative gap between the best known search node and the
incumbent; see Section~\ref{sec:pthread}.  You are free to override
this method with something more specific to your application, but the
details are omitted here.


\subsubsection{The ramp-up phase and parallel preprocessing}
\label{sec:rampupdetails}
Section~\ref{sec:rampup} describes how PEBBL can take advantage of
non-tree parallelism during the early growth of the search tree.
During the ramp-up phase, the methods \texttt{boundComputation},
\texttt{splitComputation}, and \texttt{makeChild} are called
synchronously on identical subproblems for all processors.  In your
\texttt{\emph{myParSub}} class, you may further override the
implementations of these methods in your \texttt{\emph{myBranchSub}}
class, so they can exploit synchronous parallelism during ramp-up.
The method \texttt{parallelBranchSub::rampingUp()} will return
\texttt{true} during the ramp-up phase, and \texttt{false} otherwise.

Just before the ramp-up phase starts, but after the problem
description has been broadcast, all processors invoke
\texttt{preprocess()} and \texttt{initialGuess()}.  The default behavior
of \texttt{preprocess} is to just have all processors redundantly call
the serial \texttt{preprocess()} method for the application.  If you
override \texttt{preprocess()} in your parallel-layer classes, however,
you may substitute a different computation, perhaps parallelizing your
preprocessing calculations.

A common source of errors during the ramp-up phase is processors 
not being sufficiently synchronized.  If, for example, two processors do
not agree on the value of the incumbent, they may start following
different computational paths, eventually resulting in an MPI
deadlock.  At any point in your implementation of ramp-up that you
could change the incumbent in a way that might not be identical for
all processors, you should call
\begin{codeblock}
\texttt{parallelBranchSub::rampUpIncumbentSync()}
\end{codeblock}
before taking any action that could lead to a different pattern of
search tree exploration or any variations in MPI routine calls.
This method forcibly synchronizes the incumbent and (if enumeration is
active) the solution repository.  In parallel settings, PEBBL
automatically calls \texttt{rampUpIncumbentSync()} immediately after
\texttt{initialGuess}, so your \texttt{initialGuess} routine may
return different solutions in different processors without any special
subsequent action.  PEBBL will select one of the best solutions
as the incumbent; if enumeration is active, other solutions may be
retained in the repository.

The \texttt{foundSolution} method has a second, optional argument
\texttt{sync} which is relevant only during the ramp-up phase, and
only if enumeration is active.  Its
default value is the \texttt{enum} constant \texttt{notSynchronous}.
Instead supplying the value \texttt{synchronous} indicates to PEBBL
that you are guaranteeing that all processors are calling the routine
at the same time, with the first argument set to identical solutions.
This information allows PEBBL to reduce the overhead associated with
maintaining the solution repository.

The following methods are useful in controlling the ramp-up phase.

\method{void parallelBranchSub::rampUpIncumbentHeuristic()} 
This method substitutes for the usual
\texttt{quickIncumbentHeuristic()} during the ramp-up phase. The
default implementation of \texttt{rampUpIncumbentHeuristic()} is:
\begin{codeblock}
      if \=(bG\=lobal()->haveIncumbentHeuristic()) \\
\>	\{ \\
\>\>	  incumbentHeuristic(); \\
\>\>	  pGlobal()->rampUpIncumbentSync();\\
\>      \}
\end{codeblock}
The effect of this code is to run the \emph{serial} incumbent
heuristic and then synchronize across processors.  If you
like, you may have your implementation of
\texttt{rampUpIncumbentHeuristic()} call
\texttt{feedToIncumbentThread()}, even though the incumbent heuristic
thread will not be running yet.  This practice allows the incumbent
thread on each processor to be ``primed'' with information found
during the ramp-up phase, so it may immediately begin doing useful
work once ramp up ends.

\method{bool parallelBranching::continueRampUp()} 
Ramp up will
continue as long as either this method or
\texttt{forceContinueRampUp()} return \texttt{true}.  The default
implementation is described in Section~\ref{sec:rampup}, but you are
free to override it.

\method{bool forceContinueRampUp()}
Ramp up will
continue as long as either this method or
\texttt{continueRampUp()} return \texttt{true}.  The default
implementation is described in Section~\ref{sec:rampup}, but you are
free to override it.

\method{void rampUpCleanUp()}
PEBBL calls this method on all processors when the ramp-up phase is
over.  The default implementation is a stub.


\subsubsection{Providing for checkpoints}
If your application maintains data structures not written and read by
your implementations of the methods
\texttt{\emph{myParBranching}::\linebreak[0]pack},
\texttt{\emph{myParBranching}::\linebreak[0]unpack},
\texttt{\emph{myParSub}::\linebreak[0]pack}, and
\texttt{\emph{myParSub}::\linebreak[0]unpack}, then PEBBL's
checkpointing feature will neither save nor restore them.  To make
sure any such information is saved with each checkpoint and restored
whenever restarting from a checkpoint, 
you need to define a few extra methods:

\method{void \emph{myParBranching}::appCheckpointWrite(PackBuffer\&
  outBuf)}
Write application-specific data to the \texttt{PackBuffer}
  \texttt{outBuf}, typically using the \texttt{<<} operator.  This method
  will be called separately for each processor and for each
  checkpoint.  The default implementation is a stub.

\method{void \emph{myParBranching}::appCheckpointRead(UnPackBuffer\& inBuf)}
Read the information written by \texttt{appCheckpointWrite} from
\texttt{inBuf}, typically using the \texttt{>>} operator.  This
routine will be called on each processor when a checkpoint is read
using the \texttt{restart} option.  The default implementation is a stub.

\method{void \emph{myParBranching}::appMergeGlobalData(UnPackBuffer\&
inBuf)} 
This method is similar to the \texttt{appCheckpointRead}
routine, but invoked when restarting with \texttt{--reconfigure}.  It
will be called on every processor, but multiple times --- once for
each dataset written by \texttt{appCheckpointWrite} when the
checkpoint was created.  When restarting with \texttt{--reconfigure},
the numbers of worker and hub processors may be different from when the
checkpoint was written.

\subsubsection{The main program: serial/parallel drivers}

%% \begin{figure}[tbp]
%% \begin{codeblock}
%% int \=main(int argc, char* argv[])\\
%% \{ \\
%% \>  bool flag = true; \\
%% \\
%% \>  InitializeTiming(); \\
%%  \\
%% \>  uMPI::init(\&argc,\&argv,MPI\_COMM\_WORLD); \\
%% \>  CommonIO::begin(); \\
%% \>  CommonIO::setIOFlush(1); \\
%% \\
%% \>  myParBranching instance; \\
%% \>  flag = instance.setup(argc,argv); \\
%% \>  if \=(fl\=ag) \\
%% \>\>	    \{ \\
%% \>\>\>	      instance.reset(); \\
%% \>\>\>        instance.printConfiguration(); \\
%% \>\>\>	      instance.solve(); \\
%% \>\>	    \} \\
%%  \\
%% \>	CommonIO::end(); \\
%% \>      uMPI::done(); \\
%%  \\
%% \>  return !flag; \\
%% \}
%% \end{codeblock} 
%% \caption{Example parallel driver program.}
%% \label{fig:pardriver}
%% \end{figure}

PEBBL provides a template that constructs a driver program similar to
that of Section~\ref{sec:serialdriver}, but able to sense and use
available parallelism.  If PEBBL senses that more than one MPI
processor is available or if the
command-line option \texttt{--forceParallel} is present, it invokes
the parallel-layer version of your application; otherwise, it invokes
the serial-layer version.  If it invokes the serial-layer version,
only the command-line parameters defined for that version will be
recognized; if you have defined additional parameters in your
parallel-layer classes, they will only be recognized if that layer is
invoked. 

The serial/parallel driver template is also called
\texttt{driver}, and its syntax is nearly identical to the serial
\texttt{driver} template, but with two template arguments
\texttt{<\emph{B},\emph{PB}>}.  \texttt{\emph{B}} refers to your serial
\texttt{branching}-derived problem-instance class, and \texttt{\emph{PB}}
refers to your \texttt{parallelBranching}-derived class.  For example,
you could the template invoke via
\begin{codeblock}
driver<\emph{myBranching},\emph{myParBranching}>(argc,argv)
\end{codeblock}

If you have configured PEBBL without MPI, then the two-argument form
of \texttt{driver} is still defined, but the second template argument
\texttt{\emph{PB}} is ignored.  Thus, you can use a single driver program
for your application whether or not you have configured PEBBL to use
MPI.  Once it has detected whether multiple MPI processors are available
and whether \texttt{--forceParallel} was used, the \texttt{driver}
template invokes, in both the serial and parallel cases, essentially
the same \texttt{setup}-\texttt{reset}-\texttt{solve} code pattern as
described in Section~\ref{sec:serialdriver}.

\begin{figure}[tbp]
\begin{center}
\fbox{
\begin{minipage}{0.96\textwidth}
\vspace{1ex}
\begin{codeblock}
\textrm{\emph{Include application-specific header files, for instance}}
\emph{myParBranching.h} \\
\textrm{\emph{Include~}}<pebbl/parBranching.h>\textrm{\emph{, unless
    already included in preceding headers}} \\
\\
\#ifndef ACRO\_HAVE\_MPI \\
typedef void \emph{myParBranching}; \\
\#endif \\
\\
int main(int argc, char* argv[]) \\
\{\\
~~~return driver<\emph{myBranching},\emph{myParBranching}>(argc,argv); \\
\}
\end{codeblock}
\vspace{1ex}
\end{minipage}
}
\end{center}
\vspace{-2ex}
\caption{Standard code pattern for creating a serial/parallel driver program.}
\label{fig:pardriver}
\end{figure}

The standard code pattern for creating a serial/parallel driver is
shown in Figure~\ref{fig:pardriver}.  The \texttt{\#ifndef} block is
needed if you have enclosed the definition of
\texttt{\emph{myParBranching}} within an \texttt{\#ifdef
  ACRO\_HAVE\_MPI} block.  In this case, even though the second
template argument \texttt{\emph{myParBranching}} will be ignored in
the instantiation of \texttt{driver}, one must make
\texttt{\emph{myParBranching}} a defined type to guard against
possible compiler errors; with some compilers, this safeguard may not
be needed.

Depending on your MPI environment, your serial/parallel driver may
have to be invoked in a particular way in order to detect the
availability of parallelism.  The most common such invocation
procedure involves the \texttt{mpirun} command, as specified in the
MPI standard.  For example,
\begin{codeblock}
mpirun -np 4 \emph{myParDriver} --useIncumbentThread=false 
\textrm{\emph{datafile}}
\end{codeblock}
would run \texttt{\emph{myParDriver}} on four processors, without an
incumbent heuristic thread, and with the input file \emph{datafile}.
See Section~\ref{sec:pthread} for a description of the
\texttt{useIncumbentThread} option.

Note, however, that some installations may use different, nonstandard
commands for running MPI programs.

\subsubsection{Retrieving solutions to C++ code}
\label{sec:pargetsol}
As in the serial case, you may invoke PEBBL search as a subroutine
from your own parallel C++ applications.  The methods available for
retrieving solutions are identical to the serial layer methods described is
Section~\ref{sec:serialgetsol} --- that is, \texttt{getSolution},
\texttt{getAllSolutions}, \texttt{startRepositoryScan}, and
\texttt{nextRepositoryMember} --- but the parallel layer redefines
them to work properly in parallel.  It is critical that these routines
be called synchronously on all processors, or the PEBBL application
will probably hang.

The solution-retrieval methods \texttt{getSolution},
\texttt{getAllSolutions}, \texttt{startRepositoryScan}, and
\texttt{nextRepositoryMember} have an optional argument that is
ignored in the serial layer, specifying which processor(s) the result(s)
should be returned to.  The default for this argument is the special value 
\texttt{pebblBase::allProcessors}, specifying that results be returned
to all processors through a broadcast operation.  Otherwise, the
argument should be set to an integer $k$ in the range $0 \leq k <
\text{\texttt{uMPI::rank}}$ specifying the particular processor that
is to receive the solution(s).  On other processors,
\texttt{getAllSolutions} will leave its argument unmodified, and
\texttt{getSolution} and \texttt{nextRepositoryMember} will return
\texttt{NULL}.

If the repository is very large, it is possible that it will fit
in the memory of the entire processor ensemble, but not in the memory
of a single processor.  In this case, using \texttt{getAllSolutions}
will cause out-of-memory errors, and it is preferable to use the
iterator-like methods \texttt{startRepositoryScan} and
\texttt{nextRepositoryMember}, so that the entire repository need not
reside simultaneously within a single processor.  

As in the serial case, when a solution returned by \texttt{getSolution},
\texttt{getAllSolutions}, or
\texttt{nextRepositoryMember} is no longer needed, one should call its
\texttt{dispose()} method in order to enable proper reference counting and
avoid memory leaks.

As an example, assuming that \texttt{\emph{instance}} is an object
whose type is derived from the \texttt{parallelBranching} class, and
that the \texttt{search()} method has already run to completion, the
code in Figure~\ref{fig:parreposscan} sends every solution in the
repository to processor $0$, which prints their objective values.

\begin{figure}[tbp]
\begin{center}
\fbox{
\begin{minipage}{0.96\textwidth}
\vspace{1ex}
\begin{codeblock}
int nSols = \emph{instance}.startRepositoryScan(); \\
for\= (int i=0; i<nSols; i++) \\
   \>\{~~\=\\
   \>\>solution* solPtr = \emph{instance}.nextRepositoryMember(0); \\
   \>\>if \=(uMPI::rank == 0) \\
   \>\>\>\{~~\=\\
   \>\>\>\>ucout << solPtr->value << endl; \\
   \>\>\>\>solPtr->dispose(); \\
   \>\>\>\}\\
   \>\}
\end{codeblock}
\vspace{0.1ex}
\end{minipage}
}
\end{center}
\vspace{-2ex}
\caption{Example of scanning the solution repository in the parallel layer.}
\label{fig:parreposscan}
\end{figure}

